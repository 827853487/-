%请使用XeLaTeX进行编译!!!
\documentclass{beamer}
\usepackage{ctex, hyperref}
%\usepackage[T1]{fontenc}
% other packages
\usepackage{latexsym,amsmath,xcolor,multicol,booktabs,calligra}
\usepackage{graphicx,pstricks,listings,stackengine}
% 注释功能
\usepackage{marginnote}
\usepackage{xeCJK}
%\usepackage{ctex}
%拼音功能
\usepackage{xpinyin}
\newfontfamily\PinYinFont{TeX Gyre Adventor}
\xpinyinsetup{font=\PinYinFont}
\xeCJKsetup{AutoFakeBold=true}
%图片功能
\usepackage{caption}
\setbeamerfont{caption}{size=\footnotesize, family=\kaishu}  % 设置字体大小和楷书
\captionsetup[figure]{labelformat=empty}  % 隐藏"图X"序号
% \tiny, \scriptsize, \footnotesize, \small, \normalsize(默认)
% \large, \Large, \LARGE, \huge, \Huge


%封面信息
\author{李嘉图\textrm{}}
\title{《道德经》系列读书会\textrm{}} %小括号内/右可填写内容,罗马字体
\subtitle{第一期\textrm{}}
\institute[陀思妥耶夫斯基读书会]{陀门中哲分部} % 方括号内是简称,外面是全称
\date{\textrm{2025}年\textrm{4}月\textrm{6}日}
\usepackage{SudaBeamer}

%设置缩进
\usepackage{indentfirst}

%设置颜色
\def\cmd#1{\texttt{\color{red}\footnotesize $\backslash$#1}}
\def\env#1{\texttt{\color{blue}\footnotesize #1}}
\definecolor{deepblue}{rgb}{0,0,0.5}
\definecolor{deepred}{rgb}{0.6,0,0}
\definecolor{deepgreen}{rgb}{0,0.5,0}
\definecolor{halfgray}{gray}{0.55}

% 定义文言文相关颜色
\definecolor{classical}{RGB}{89,61,43}     % 文言文颜色:深褐色
\definecolor{trans}{RGB}{76,120,40}        % 译文颜色:墨绿色
\definecolor{notemark}{RGB}{150,50,30}     % 注释颜色:暗红色
\definecolor{comment}{RGB}{44,67,88}       % 评论颜色:藏青色

% 定义文言文的文本样式
\newcommand{\classical}[1]{\textcolor{classical}{\kaishu #1}}         % 文言文
\newcommand{\trans}[1]{\textcolor{trans}{\fangsong #1}}              % 译文
\newcommand{\notation}[1]{\textsuperscript{\textcolor{notemark}{\small #1}}} % 注释
\newcommand{\comment}[1]{\textcolor{comment}{\kaishu #1}}            % 评论

% 定义文言文边注命令
\newcommand{\sidenote}[1]{\marginnote{\textcolor{notemark}{\small #1}}}

% 定义文本框
\newcommand{\classicalbox}[1]{
	\begin{beamercolorbox}[wd=\textwidth,sep=1em,rounded=true]{block title}
		\classical{#1}
	\end{beamercolorbox}
	}
\newcommand{\transbox}[1]{
	\begin{beamercolorbox}[wd=\textwidth,sep=1em,rounded=true]{block body}
		\trans{#1}
	\end{beamercolorbox}
	}

% 定义原文展示框
\newcommand{\originaltext}[1]{
	\begin{center}
		\large\classical{#1}
	\end{center}
	}

% 定义字体与模板混色
\lstset{
	basicstyle=\ttfamily\small,
	keywordstyle=\bfseries\color{deepblue},
	emphstyle=\ttfamily\color{deepred},    % Custom highlighting style
	stringstyle=\color{deepgreen},
	numbers=left,
	numberstyle=\small\color{halfgray},
	rulesepcolor=\color{red!20!green!20!blue!20},
	frame=shadowbox,
	}


\begin{document}

\kaishu
% \setmainfont{Times New Roman}
\begin{frame}
	\titlepage 
	\begin{figure}[htpb]
		\begin{center}
			\includegraphics[width=0.30\linewidth]{pic/tuomen.png}
		\end{center}
	\end{figure}
\end{frame}

\begin{frame}
	\tableofcontents
\end{frame}

%%%%%%%%%%%%%%%%%%%%%%%%%%% 分隔符 %%%%%%%%%%%%%%%%%%%%%%%%%%%%%

\section{老子与《道德经》}
\begin{frame}{老子生平}
    \begin{columns}
		\column{0.4\textwidth}
        \begin{center}  % 使整个图片和标题居中
            \includegraphics[width=0.9\textwidth]{pic/laozi.png}
            \captionof{figure}{\centering 老子\\(约公元前571年-前471年)}
        \end{center}
		\column{0.6\textwidth}
            姓李名耳,字伯阳,谥号聃 \\
%==============================%
		\vspace{0.2 cm}
%==============================%
            - 春秋时期楚国苦县人(今河南鹿邑)\\
            - 曾任周朝“守藏室之史” \\
            - 晚年隐居,西出函谷关著《道德经》
    \end{columns}
\end{frame}

\begin{frame}{《道德经》介绍}
		\begin{minipage}{0.3\textwidth}
			\centering
  			\includegraphics[width=0.665\linewidth]{pic/banboo.png}
  			\captionof{figure}{竹简本}
		\end{minipage}
		\hfill
		\begin{minipage}{0.3\textwidth}
			\centering
			\includegraphics[width=0.51\linewidth]{pic/boshu.png}
			\captionof{figure}{帛书本}
		\end{minipage}
		\hfill
		\begin{minipage}{0.3\textwidth}
			\centering
			\includegraphics[width=0.7\linewidth]{pic/wangbi.png}
			\captionof{figure}{王弼《老子注》}
		\end{minipage}
		
%==============================%
%    	\vspace{0.1 cm}
%==============================% 

	\begin{itemize}
		\item 成书时间被认为在春秋末期或战国初期 %(约公元前485年)
		\item 郭店楚墓出土的\alert{竹简《老子》}抄录于战国中晚期 % 出土于湖北荆门(约公元前300年)
		\item 马王堆汉墓出土的\alert{帛书甲乙本}抄录于汉文帝时期 % 出土于湖南长沙(约公元前206年至公元前180年)
		\item 继西汉河上公注本后,通行本\alert{王弼注本}成书于三国时期 % 约公元226年至公元249年
	\end{itemize}
\end{frame}

\begin{frame}{“道”与“德”}
%    \vfill
    \begin{block}{何谓“道”}
    	\begin{itemize}
    		\item 在金文中作“行”中置“首”,初指道路与方向。
			\item “道”是天地间所有\alert{规律、真理}的总称。
			\item 代指人或物所必须遵循的客观法则。
		\end{itemize}
    \end{block}
    \vfill
    
    \begin{block}{何谓“德”}
		\begin{itemize}
			\item 在甲骨文中与“直”同源,本义为“目视标杆以正行路”。
			\item “德”是\alert{具体事物}的规律、本性。
			\item 已经被人掌握的这一部分道就叫作“德”。
		\end{itemize}
    \end{block}
    \vfill
\end{frame}

%%%%%%%%%%%%%%%%%%%%%%%%%%% 分隔符 %%%%%%%%%%%%%%%%%%%%%%%%%%%%%

% 普通文本页示例
%\begin{frame}{普通文本示例}
%    这是一个普通的文本页面。
%    \begin{block}{普通 Block}
%        这里是一个 Beamer 的 block。
%    \end{block}
%    \begin{alertblock}{警告 Block}
%        这里是一个 alert block,用于高亮重要信息。
%    \end{alertblock}
%    \begin{exampleblock}{示例 Block}
%        这里是一个 example block,适用于示例展示。
%    \end{exampleblock}
%\end{frame}

%%%%%%%%%%%%%%%%%%%%%%%%%%% 分隔符 %%%%%%%%%%%%%%%%%%%%%%%%%%%%%

\section{原文解读}
\begin{frame}{原文解读}
	\begin{center}
		{\large\calligra 上篇·道经}
	\end{center}
\end{frame}

\subsection{第一章}
%原文及注释
\begin{frame}{第一章·注释}
	\originaltext{
	道可道,非常道\notation{1}。\\
	名可名,非常名\notation{2}。\\
	无名\notation{3},天地之始。有名\notation{4},万物之母。\\
	故常无欲,以观其妙\notation{5};常有欲,以观其\xpinyin{徼}{jiao4}\notation{6}。\\
	此两者同出而异名,同谓之玄\notation{7}。\\
	玄之又玄,众妙之门\notation{8}。
	}
	\begin{itemize}
		\item[\notation{1}] 第一个“道”为名词,第二个为动词,下一句同。
		\item[\notation{2}] 第一个“名”指具体事物的名称。第三个称“道”之名。
		\item[\notation{3}] 无名:即“无”。指未分化的混沌状态。
		\item[\notation{4}] 有名:指构成万物的最基本的物质元素。
	\end{itemize}
\end{frame}
%加5/6/7/8注释的话分第二页摆放
%注释续
\begin{frame}{第一章·注释}
	\originaltext{
	道可道,非常道\notation{1}。\\
	名可名,非常名\notation{2}。\\
	无名\notation{3},天地之始。有名\notation{4},万物之母。\\
	故常无欲,以观其妙\notation{5};常有欲,以观其\xpinyin{徼}{jiao4}\notation{6}。\\
	此两者同出而异名,同谓之玄\notation{7}。\\
	玄之又玄,众妙之门\notation{8}。
	}
	\begin{itemize}
		\item[\notation{5}] 第一个“道”为名词,第二个为动词。
		\item[\notation{6}] 徼(jiào):边界。这里引申为表象。
		\item[\notation{7}] 玄者,冥也,默然无有也。(王弼注)
		\item[\notation{8}] 玄:幽昧深远,形容道的深邃。
	\end{itemize}
\end{frame}

%原文翻译
\begin{frame}{第一章·译文}
	\originaltext{
	道可道,非常道。名可名,非常名。\\
	无名,天地之始。有名,万物之母。\\
	故常无欲,以观其妙;常有欲,以观其徼。\\
	此两者同出而异名,同谓之玄。玄之又玄,众妙之门。
	}
	\begin{columns}[T]
%		\column{0.45\textwidth}
		\column{1.0\textwidth}
		\trans{\setlength{\parindent}{2em}\\
		%\hspace{1.0em}
		可以用语言描述清楚的道,就不是永恒不变的大道;能够用来称呼的具体名称,就不是永恒不变的名称。\\
		空虚无名的空间,是天地得以出现的开始;真实有名的基本物质,是万物得以产生的根源。\\
		因此如果一个人能够经常保持清静无欲的心态,就可以观察空间和万物的微妙之处;如果经常处于多欲状态,就只能看到空间和万物的一些表面现象。\\
		空间与物质同时出现而有不同的名称,它们可以说都是非常奥妙的。如果能够反复不断地去探索它们的奥妙,那么就能够打开通向天地万物奥秘的大门。
		}
		
%		\column{0.45\textwidth}
%		\comment{本章开篇即点明"道"的玄妙难言,"名"的局限性,以及有无相生的辩证关系。}
	\end{columns}
\end{frame}

%章节赏析
\begin{frame}{第一章·解读}
	\originaltext{
	道可道,非常道。名可名,非常名。\\
	无名,天地之始。有名,万物之母。\\
	故常无欲,以观其妙;常有欲,以观其徼。\\
	此两者同出而异名,同谓之玄。玄之又玄,众妙之门。
	}
	\begin{columns}[T]
		\column{1.0\textwidth}
		\comment{\setlength{\parindent}{2em}\\
		“道”作为所有规律的总称,其内涵丰富深奥、微妙复杂,所以很难用语言表达清楚。无论是最高真理,还是普通技能,都需要学习者亲自去体悟与实践,仅仅靠书本、语言,无法掌握其中的奥妙。\\
		在老子看来,天地万物作为“有(物质)”,是在与“无(空间)”相对立中显现的。“有无相生”这一辩证思想,贯穿了《道德经》全书。
		}
	\end{columns}
\end{frame}

%\begin{frame}{字词注音}
%	\begin{pinyinscope}
%列位看官:你道此书从何而来?说起根由,虽近荒唐,细按则深有趣味。
%待在下将此来历注明,方使阅者\xpinyin{了}{liao3}然不惑。
%	\end{pinyinscope}
%\end{frame}

\subsection{第二章}
%原文及注释
\begin{frame}{第二章·注释}
	\originaltext{
	天下皆知美之为美,斯恶已\notation{1};皆知善之为善,斯不善已。\\
	故有无相生\notation{2},难易相成,长短相较\notation{3},高下相倾\notation{4},音声相和\notation{5},前后相随。\\
	是以圣人处无为之事\notation{6},行不言之教。\\
	万物作焉而不辞\notation{7},生而不有,为而不恃\notation{8},功成而弗居。\\夫唯弗居,是以不去\notation{9}。
	}
	\begin{itemize}
		\item[\notation{1}] 恶:丑。此处用作动词,显露出丑陋。
		\item[\notation{2}] 有:物质存在。无:空间。
		\item[\notation{3}] 相较:在相互比较中存在。
		\item[\notation{4}] 相倾:相互依赖。
		\item[\notation{5}] 经过修饰的声音叫作“音”,自然而然发出的声音叫作“声”。
	\end{itemize}
\end{frame}
%注释续
\begin{frame}{第二章·注释}
	\originaltext{
	天下皆知美之为美,斯恶已\notation{1};皆知善之为善,斯不善已。\\
	故有无相生\notation{2},难易相成,长短相较\notation{3},高下相倾\notation{4},音声相和\notation{5},前后相随。\\
	是以圣人处无为之事\notation{6},行不言之教。\\
	万物作焉而不辞\notation{7},生而不有,为而不恃\notation{8},功成而弗居。\\夫唯弗居,是以不去\notation{9}。
	}
	\begin{itemize}
		\item[\notation{6}] 处:行,做。无为:顺应自然而为。
		\item[\notation{7}] 作:兴起,出现。不辞:不拒绝,不限制。
		\item[\notation{8}] 恃:依赖。引申为追求回报。
		\item[\notation{9}] 是以不去:因此不会失去。
	\end{itemize}
\end{frame}

%原文翻译
\begin{frame}{第二章·翻译}
	% 保持原文在上方
	\originaltext{
	天下皆知美之为美,斯恶已;皆知善之为善,斯不善已。\\
	故有无相生,难易相成,长短相较,高下相倾,音声相和,前后相随。\\
	是以圣人处无为之事,行不言之教。\\
	万物作焉而不辞,生而不有,为而不恃,功成而弗居。\\夫唯弗居,是以不去。
	}
	\begin{columns}[T]
		\column{1.10\textwidth}
		\trans{\setlength{\parindent}{2em}\\
		如果天下的人都知道美好的东西是美好的话,那么丑陋的东西就显露出来了;\\
		如果都知道善良的事情是善良的话,那么不善良的事情就显露出来了。\\
		有和无在相互对立中得以形成,难和易在相互对应中得以产生,长和短在相互比较中得以显现,高和下在相互依赖中得以存在,音和声在相互应和中得以区分,前和后在相互对比中得以出现。
		}
	\end{columns}
\end{frame}
\begin{frame}{第二章·译文}
	\originaltext{
	天下皆知美之为美,斯恶已;皆知善之为善,斯不善已。\\
	故有无相生,难易相成,长短相较,高下相倾,音声相和,前后相随。\\
	是以圣人处无为之事,行不言之教。\\
	万物作焉而不辞,生而不有,为而不恃,功成而弗居。\\夫唯弗居,是以不去。
	}
	\begin{columns}[T]
		\column{1.0\textwidth}
		\trans{\setlength{\parindent}{2em}\\
		因此圣人所做的事情就是顺应自然而不提倡人为的干涉,圣人推行的是不用语言的教育。\\
		圣人顺应万物的生长而不加以限制,生养了万物而不据为己有,帮助了万物而不要求它们的回报,建立了功劳而不据为己有。正因为圣人从不居功,所以也不会失去自己的功劳。
		}
	\end{columns}
\end{frame}

%章节赏析
\begin{frame}{第二章·解读}
	\originaltext{
	天下皆知美之为美,斯恶已;皆知善之为善,斯不善已。\\
	故有无相生,难易相成,长短相较,高下相倾,音声相和,前后相随。\\
	是以圣人处无为之事,行不言之教。\\
	万物作焉而不辞,生而不有,为而不恃,功成而弗居。\\夫唯弗居,是以不去。
	}
	\begin{columns}[T]
		\column{1.0\textwidth}
		\comment{\setlength{\parindent}{2em}\\
		相互对立的东西必须相互依赖才能存在。既然没有“无”就没有“有”,那么没有“无为”,自然也就没有“无不为”;没有占有,自然也就没有失去。\\
		由“不言之教”联想到《庄子·德充符》中的兀者王骀。“鲁有兀者王骀,从之游者与仲尼相若。……\ 立不教,坐不议,虚而往,实而归。固有不言之教、无形而心成者邪?”
		}
	\end{columns}
\end{frame}


\subsection{第三章}
%原文及注释
\begin{frame}{第三章·注释}
	\originaltext{
	不尚贤\notation{1},使民不争;\\
	不贵难得之货\notation{2},使民不为盗;\\
	不\xpinyin{见}{xian4}可欲\notation{3},使民心不乱。\\
	是以圣人之治:虚其心\notation{4},实其腹;弱其志\notation{5},强其骨。\\
	常使民无知无欲,使夫智者不敢为也\notation{6}。\\
	为无为\notation{7},则无不治。\\
    }
    
	\begin{itemize}
		\item[\notation{1}] 尚:崇尚,重用。国家“尚贤”则民众好名。
		\item[\notation{2}] 不看重金银珠玉之类的财物。
		\item[\notation{3}] 见(xiàn):同“现”,显露。可欲,即上文要破除的名利。
		\item[\notation{4}] 使人们的内心虚静而无太多的欲望和杂念。
		\item[\notation{5}] 志:志向。这里泛指欲望。
	\end{itemize}
\end{frame}
%注释续
\begin{frame}{第三章·注释}
	\originaltext{
	不尚贤\notation{1},使民不争;\\
	不贵难得之货\notation{2},使民不为盗;\\
	不\xpinyin{见}{xian4}可欲\notation{3},使民心不乱。\\
	是以圣人之治:虚其心\notation{4},实其腹;弱其志\notation{5},强其骨。\\
	常使民无知无欲,使夫智者不敢为也\notation{6,7}。\\
	为无为\notation{8},则无不治。\\
    }
    
	\begin{itemize}
		\item[\notation{6}] 指世俗社会所认为的聪明人,而非真正的智者。
		\item[\notation{7}] 为:与“无为”相对,指按照主观意愿行事。
		\item[\notation{8}] 执行“无为”的政策。
	\end{itemize}
\end{frame}

%原文翻译
\begin{frame}{第三章·译文}
	\originaltext{
	不尚贤,使民不争;\\
	不贵难得之货,使民不为盗;\\
	不见可欲,使民心不乱。\\
	是以圣人之治:虚其心,实其腹;弱其志,强其骨。\\
	常使民无知无欲,使夫智者不敢为也。\\
	为无为,则无不治。
	}
	\begin{columns}[T]
		\column{1.1\textwidth}
		\trans{\setlength{\parindent}{2em}\\
		不要崇尚、重用贤人,百姓就不会去争夺美名;\\
		不要看重贵重难得的金银财宝,百姓就不会去做盗贼;\\
		不显露那些可以引起人们欲望的事物,百姓的心就不会被搅乱。\\
		因此圣人治国的原则是:减少百姓的杂念,填饱他们的肚皮;降低百姓的欲望,增强他们的体质。\\
		永远使百姓没有多少知识和欲望,使那些所谓的聪明人不敢按照主观意愿去为所欲为。\\
		执行无为的政策,天下就会安定太平。
		}
	\end{columns}
\end{frame}

%章节赏析
\begin{frame}{第三章·解读}
	\originaltext{
	不尚贤,使民不争;\\
	不贵难得之货,使民不为盗;\\
	不见可欲,使民心不乱。\\
	是以圣人之治:虚其心,实其腹;弱其志,强其骨。\\
	常使民无知无欲,使夫智者不敢为也。\\
	为无为,则无不治。
	}
	\begin{columns}[T]
		\column{1.0\textwidth}
		\comment{\setlength{\parindent}{2em}\\
		老子主张破除人们的好名、好利之心。认为一旦君主提倡重用贤人,真正的贤人未必就去出仕,而那些不贤的人为了名利权势而投君主所好,假扮成贤人的模样,获得权力后危害国家和百姓。\\
		“虚心”之人,随遇而安,无可无不可,高官厚禄无法使他欣喜,穷困潦倒无法使他悲伤,命运无法给他带来丝毫的痛苦。\\
		无奈命何,委顺以待终。命无奈我何,方寸如虚空。(白居易《达理二首·其二》)
		}
	\end{columns}
\end{frame}


\subsection{第四章}
%原文及注释
\begin{frame}{第四章·注释}
	\originaltext{
	道冲\notation{1},而用之或不盈\notation{2}。\\
	渊兮,似万物之宗\notation{3}:\\
	挫其锐,解其纷\notation{4};和其光,同其尘\notation{5}。\\
	湛兮,似或存\notation{6}。\\
	吾不知谁之子,象帝之先\notation{7}。
	}
	\begin{itemize}
		\item[\notation{1}] 冲:空虚,无形无象。一说同“盅”,器虚也。
		\item[\notation{2}] 而:如果。用:遵循。盈:盈满。
		\item[\notation{3}] 大道如深渊般深邃奥妙,难以认识。宗:宗主,主宰者。
		\item[\notation{4}] 挫去万物的锋芒,解除世间的纷扰。
		\item[\notation{5}] 指不露锋芒,与世无争。
	\end{itemize}
\end{frame}
%注释续
\begin{frame}{第四章·注释}
	\originaltext{
	道冲\notation{1},而用之或不盈\notation{2}。\\
	渊兮,似万物之宗\notation{3}:\\
	挫其锐,解其纷\notation{4};和其光,同其尘\notation{5}。\\
	湛兮,似或存\notation{6,7}。\\
	吾不知谁之子,象帝之先\notation{8}。
	}
	\begin{itemize}
		\item[\notation{6}] 湛:无形无象、看不见摸不着的样子。
		\item[\notation{7}] 似或:好像。或:也许,似乎。
		\item[\notation{8}] 象者,有形之始也;帝者,生物之祖也。(王安石注)
    \end{itemize}
\end{frame}

%原文翻译
\begin{frame}{第四章·译文}
	\originaltext{
	道冲,而用之或不盈。\\
	渊兮,似万物之宗:\\
	挫其锐,解其纷;和其光,同其尘。\\
	湛兮,似或存。\\
	吾不知谁之子,象帝之先。
	}
	\begin{columns}[T]
		\column{1.0\textwidth}
		\trans{\setlength{\parindent}{2em}\\
		道是无形无象的,如果遵循它办事,也许就不会要求把事情办到盈满、极盛的状态。\\
		大道就像深渊那样深邃奥妙而难以认识,它好像是万物的主宰者:\\
		它挫去万物的锋芒,从而化解它们之间的纠纷;调和它们的优点,从而使它们都有一定的缺陷。\\
		道是无形无象的,但似乎确实存在着。\\
		我不知道是谁产生了它,只知道它出现于万物之前。
		}
	\end{columns}
\end{frame}

%章节赏析
\begin{frame}{第四章·解读}
	\originaltext{
	道冲,而用之或不盈。\\
	渊兮,似万物之宗:\\
	挫其锐,解其纷;和其光,同其尘。\\
	湛兮,似或存。\\
	吾不知谁之子,象帝之先。
	}
	\begin{columns}[T]
		\column{1.0\textwidth}
		\comment{\setlength{\parindent}{2em}\\
		“日极则仄,月满则亏”,做人做事不能过分追求盈满,而要留有一定的余地;\\
		“木秀于林,风必摧之”,老子建议人们以“和光同尘”的保身原则处世;\\
		老子认为“道”先于世间万物而存在,处于宇宙的本源地位。% 宇宙观
		}
	\end{columns}
\end{frame}


%%%%%%%%%%%%%%%%%%%%%%%%%%% 分隔符 %%%%%%%%%%%%%%%%%%%%%%%%%%%%%

\section{自由讨论}
\begin{frame}{讨论主题}
	\begin{enumerate} % 带序号列表
		\item \alert{“道”}与\alert{“名”}的联系与区别是什么?
		\item 怎样理解\alert{“有无相生”}的辩证思想?
		\item 老子所推崇的\alert{“无为而治”}是否等同于消极不作为?
		\item 是否应该认同\alert{“和光同尘”}的圆滑处世之方?
		\item 《道德经》包含的哲学思想属于\alert{主观唯心主义}吗?
		\item “道”应该被当作一种\alert{客观存在}来认识,还是作为一种\alert{虚无归因}来理解?
%		\item 王弼“以无为本”与河上公“治身治国”两种注释传统,哪种更贴近原意?
    \end{enumerate}
\end{frame}

%%%%%%%%%%%%%%%%%%%%%%%%%%% 分隔符 %%%%%%%%%%%%%%%%%%%%%%%%%%%%%

\section{参考资料}
\begin{frame}{参考资料}
	\begin{columns}%[T] % [T] 表示顶部对齐
    % 左侧文字栏(宽度设为40%)
	\column{0.4\textwidth}
%==============================%
%    \vspace{-3.2 cm}
%==============================%
	\centering
		\begin{center}  % 使整个图片和标题居中
			\includegraphics[width=0.86\textwidth]{pic/china.png}
			\captionof{figure}{\centering 《道德经》\\中华书局出版}
		\end{center}

    % 右侧图片栏(宽度设为60%)
	\column{0.6\textwidth}
	\begin{exampleblock}{推荐读本}
		\begin{itemize}
			\item {《道德经》(张景、张松辉\ 译注)}
		\end{itemize}
	\end{exampleblock}
    
	\begin{exampleblock}{拓展资料}
		\begin{itemize}
			\item 王弼\ \ \ 《老子道德经注》
			\item 陈鼓应《老子今注今译》
			\item 任继愈《老子绎读》
      \end{itemize}
    \end{exampleblock}
%==============================%
    	 \vspace{0.5 cm}
%==============================%
  \end{columns}
\end{frame}

\begin{frame}{书法欣赏} % 任法融 前道教协会会长
    \centering
    \includegraphics[width=0.9\textwidth]{pic/daofaziran.png}
    \captionof{figure}{\alert{任法融}书法作品}
\end{frame}

\begin{frame}
    \begin{center}
        {\Large\calligra 谢谢倾听!}
    \end{center}
\end{frame}

\end{document}